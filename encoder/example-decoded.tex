\documentclass{article}
\usepackage{epsfig}
\usepackage{hyperref}
\renewcommand{\baselinestretch}{1}
\setlength{\textheight}{9in}
\setlength{\textwidth}{6.5in}
\setlength{\headheight}{0in}
\setlength{\headsep}{0in}
\setlength{\topmargin}{0in}
\setlength{\oddsidemargin}{0in}
\setlength{\evensidemargin}{0in}
\setlength{\parindent}{.3in}
\begin{document}

\centerline {\Large \bf 26L-04 CLASS POLICY}

\gotable{0}{1}{rlp{.1in}rl}
{\bf Instructor:} & Laura Taalman 		 
   && {\bf Office:}   & 020 Math/Physics \\
{\bf Telephone:}  & 660-2829 (W), 220-1359 (H) 
   && {\bf E-mail:}   & taal@math.duke.edu \\
\stoptable

General policies for 26L are found in your Coursepack on pages 5-14.  
Policies specific to my section are listed below, and supercede any 
policies in the coursepack.  You are expected to understand and agree 
to all of these policies.  I will be unimpressed if I have to answer  
lots of questions throughout the semester whose answers can be found 
here or in the Coursepack.  
\vs

{\bf Grades}
Your grade for this course will be determined almost entirely by your 
performance on quizzes, tests, and the final exam.  I expect the 
distribution of points (and thus the corresponding percentages) to be 
as follows:
%
\gotable{2}{1.2}{|l|c|r|r|}
\hline
Weekly Quizzes & 10 @ 25 pts & 250 total pts & 31.25 \% \\
Lab Reports    & 2 @ 25 pts  & 50 total pts  &  6.25 \% \\
Tests          & 3 @ 100 pts & 300 total pts & 37.50 \% \\
Final Exam     & 1 @ 200 pts & 200 total pts & 25.00 \% \\
\hline
\stoptable

Weekly quiz scores can be modified by your score on ``attendance 
quizzes'' and your completion of ``gateway homeworks'' (where 
applicable).  Test scores can be increased by completion of cheat 
sheets.  The final exam may end up counting as more than 25 \% 
of your grade; see page 11 in your coursepack for more information.
\vs
 
{\bf Attendance and Absences:} 
It is imperative that you come to class 
each day and take careful notes.  I will take random ``attendance 
quizzes'' throughout the semester to keep track of who is attending 
class.  Each of these quizzes will be worth two extra points towards 
that week's homework/lab quiz (not to exceed 25 total points); these 
quizzes cannot be made up for any reason.  If you 
do miss class it is your responsibility to find out (from a classmate) 
what you missed, including class notes, announcements, and worksheets.  
Absences from tests will be excused only for reasons such as serious 
illness or official university activities.  In some cases the student 
can obtain an excuse from his/her Dean.  A student who misses a test 
because of illness and cannot obtain a Dean's excuse should pick up a 
departmental excuse form from the Mathematics Department Office, 121 
Physics Bldg., fill it out, and take it to Lewis Blake, the Supervisor 
of First-Year Instruction, 118 Physics Bldg.  If the absence is 
excused, the student's grade on the missed test will be determined 
either by assigning a score according to his/her performance on the 
final exam relative to other students in the class or by a make-up 
test, at my discretion.  
\vs

{\bf Homework:} 
Homework (both the reading and the exercises) should be completed by the 
next class after it is assigned. Homework will not be collected, 
although your understanding of it will be tested in each weekly quiz 
and each test. It is your responsibility to seek help on all problems that 
you cannot do. Help is available after class, in the help room or 
by appointment with me. I will not discuss homework problems in class.

\clearpage

{\bf Quizzes:} 
There will be a 25 point, 25 minute quiz at the beginning of each lab 
period (except for the first week, and the weeks where tests occur).
The quiz will cover the previous week's lab (if there was one), and 
the homework from the last week (specifically, the Wednesday and 
Friday of the previous week, and the Monday of the current week).
Keep in mind that four of these homework/lab quizzes are worth as 
much as a test!
\vs

{\bf Labs:} 
There will be a lab each Thursday, usually beginning with the 
homework/lab quiz for that week. If you are late to lab you will have 
less time to take this quiz.  You are expected to read through 
the lab in the Coursepack or book before coming to lab.
There is no specific calculator required for this class but you should
bring some form of graphing calculator to class and lab 
(see \verb|http://www.math.duke.edu/first_year/calculator.html| and
page 12 of your coursepack).
During the lab you will work in pairs (if there are an odd number of 
students there will be \underline{one} group of three); please choose 
a permanent lab partner by the second lab.
If you do not complete the lab during the lab period (as is often the 
case), you and your lab partner are expected to find time to work 
together out of class and finish the lab.  The labs will not be 
collected (although there will be two lab reports).  You will be 
tested on your understanding of the lab material on the quizzes, 
tests, and the final exam.
\vs

{\bf Tests and Cheat Sheets:} 
There will be three tests in this course, each taking an entire lab 
period.  Please see ``Absences'' above for related information.
I do not allow the use of ``cheat sheets'' during 
the exams.  However, since some other sections do allow cheat sheets, 
and you will be allowed to use a cheat sheet during the final exam, I 
encourage you to make a one-page, two-sided ``cheat sheet'' for each 
test.  I will collect these cheat sheets (\underline{before} each 
test, of course).  You can earn up to three extra points on the test 
(not to exceed 100 total points) for handing in a complete cheat sheet.  
It should be much easier for you to make your 
final exam cheat sheet if you have already made one for each test.  
\vs

{\bf The Final:}  
The final exam for this course is a ``block'' final, meaning all 
sections will take the same exam; the final will be given
on Tuesday, May 4, 1999 at 7 pm - 10 pm. See
http://registrar.duke.edu/registrar/exams99.htm and page 11 in
your Coursepack for more information. 
\vs

{\bf The Gateway:}
You will have to pass a differentiation gateway to get credit for 
this course (see page 12 in your coursepack).  Anyone who does not 
pass the gateway the first time will have to hand in a ``gateway 
homework'' in class each Friday until they have passed the gateway.  
Unsatisfactory performance on this homework will result in a deduction 
of up to five points (not to be lower than 0 total points) from your 
previous homework/lab quiz score.
\vs 

{\bf Help Room:} 
I strongly suggest that you visit the help room 
often, especially during the hours that I will be working there (to be 
announced), and while other 26L instructors or lab TAs are there.  The 
help room is in 08 West Duke (in the basement) and is open 1:00 pm - 
10:00 pm Monday - Thursday and 6:00 pm - 10:00 pm Sunday.  The help 
room will begin on Sunday, January 17, 1999, and operate through 
Sunday, May 2, except for Martin Luther King's birthday on Monday, 
January 18, and from Sunday, March 14, through Sunday, March 21, for 
spring break.  The help room is discussed briefly on page 12 of the 
Coursepack.  Follow the link on 
\verb|http://www.math.duke.edu/first_year/help.html| for more information.  

\documentclass[landscape]{article}

\oddsidemargin=-0.25in
\evensidemargin=-0.25in
\textwidth=9in
\topmargin=-.5in
\textheight=6in

\parindent=.25in
\parskip=2ex

\usepackage{amsbsy}
\usepackage{amssymb}

\input{lauracode}
\input{lauracodePHD}

\title{4 Percent of Laura's Dissertation}
\author{Laura Taalman, A.B.D.}

\begin{document}
\pagestyle{empty}

\maketitle

% here the two diagrams are together

$$\matrix{
\A    & \MapinR{\alpha}{} 
 & \B & \MapinR{}{=} 
 & \C   \cr
\mapD 
 && \mapD 
 && \MapD{\widetilde{\beta}}{\wedge\frac{dh}{h}} 
 &  \MapSE{\beta} \cr
\underbrace{\AA \vph{-2}}_{\textstyle \aa}                   & \MapinR{\alpha_2}{} 
 & \underbrace{\BB \vph{-2}}_{\textstyle \bb \tensor \O(-N)} & \mapinR 
 & \CC                                                       & \MaponR{p_2}{} 
 & \underbrace{\DD}_{\textstyle \dd}                           \cr
\mapD 
 && \mapD 
 && \MaponD{\widetilde{\gamma}}{\wedge\frac{dh}{h}} 
 && \MapD{}{\gamma}\cr                                                   
\underbrace{\AAA \vph{-2}}_{\textstyle \aaa}    & \MapinR{\approx}{\alpha_3} 
 & \underbrace{\BBB \vph{-2}}_{\textstyle \bbb} & \mapinR 
 & \underbrace{\CCC \vph{-2}}_{\textstyle \ccc} & \MaponR{p_3}{} 
 & \underbrace{\DDD}_{\textstyle \ddd}            \cr 
}$$

\pg  % below we do this as two separate diagrams

\vf

$$\matrix{
\A    & \MapinR{\alpha_1}{}        
 & \B & \mapeqR 
 & \C   \cr
\mapD 
 && \mapD 
 && \MapD{\widetilde{\beta}}{\hwedge}  
 &  \MapSE{\beta} \cr
\AA     & \MapinR{\alpha_2}{}        
 & \BB  & \mapinR      
 & \CC  & \MaponR{p_2}{} 
 & \DD    \cr
\mapD 
 && \mapD 
 && \MaponD{\widetilde{\gamma}}{\hwedge} 
 && \MapD{}{\gamma}  \cr                                                   
\AAA    & \MapinR{\alpha_3}{\approx} 
 & \BBB & \mapinR     
 & \CCC & \MaponR{p_3}{} 
 & \DDD 
}$$

\vf
{\bf a.k.a.}
\vf

$$\matrix{
\a    & \MapinR{\alpha_1}{} 
 & \b & \mapeqR 
 & \c   \cr
\mapD 
 && \mapD 
 && \MapD{\widetilde{\beta}}{\wedge\frac{dh}{h}} 
 &  \MapSE{\beta} \cr
\aa    & \MapinR{\alpha_2}{} 
 & \bb & \mapinR 
 & \cc & \MaponR{p_2}{} 
 & \dd   \cr
\mapD 
 && \mapD 
 && \MaponD{\widetilde{\gamma}}{\wedge\frac{dh}{h}} 
 && \MapD{}{\gamma} \cr                                                   
\aaa    & \MapinR{\alpha_3}{\approx} 
 & \bbb & \mapinR 
 & \ccc & \MaponR{p_3}{} 
 & \ddd
}$$

\vf
{\bf so we have}
\vf

$$\matrix{
\a      & \MapinR{\alpha_1}{}  
 & \b   & \MapRR{\beta}{\hwedge} 
 & \cc  & \MaponRRR{\gamma}{\hwedge} 
 & \ddd

\section*{Basic Font Tables}

\gotable{-1.5}{1.1}{|l|cccccccccccccccccccccccccc|}
\hline
\hline
roman      & A&B&C&D&E&F&G&H&I&J&K&L&M&N&O&P&Q&R&S&T&U&V&W&X&Y&Z \\
\verb|{\rm ...}| 
           & a&b&c&d&e&f&g&h&i&j&k&l&m&n&o&p&q&r&s&t&u&v&w&x&y&z \\
\hline
bold       & \bf A & \bf B & \bf C & \bf D & \bf E & \bf F & \bf G 
           & \bf H & \bf I & \bf J & \bf K & \bf L & \bf M & \bf N 
           & \bf O & \bf P & \bf Q & \bf R & \bf S & \bf T & \bf U 
           & \bf V & \bf W & \bf X & \bf Y & \bf Z \\
\verb|{\bf ...}| 
           & \bf a & \bf b & \bf c & \bf d & \bf e & \bf f & \bf g 
           & \bf h & \bf i & \bf j & \bf k & \bf l & \bf m & \bf n 
           & \bf o & \bf p & \bf q & \bf r & \bf s & \bf t & \bf u 
           & \bf v & \bf w & \bf x & \bf y & \bf z \\
\hline
italic     & \it A & \it B & \it C & \it D & \it E & \it F & \it G 
           & \it H & \it I & \it J & \it K & \it L & \it M & \it N 
           & \it O & \it P & \it Q & \it R & \it S & \it T & \it U 
           & \it V & \it W & \it X & \it Y & \it Z \\
\verb|{\it ...}| 
           & \it a & \it b & \it c & \it d & \it e & \it f & \it g 
           & \it h & \it i & \it j & \it k & \it l & \it m & \it n 
           & \it o & \it p & \it q & \it r & \it s & \it t & \it u 
           & \it v & \it w & \it x & \it y & \it z \\
\hline
slant      & \sl A & \sl B & \sl C & \sl D & \sl E & \sl F & \sl G 
           & \sl H & \sl I & \sl J & \sl K & \sl L & \sl M & \sl N 
           & \sl O & \sl P & \sl Q & \sl R & \sl S & \sl T & \sl U 
           & \sl V & \sl W & \sl X & \sl Y & \sl Z \\
\verb|{\sl ...}| 
           & \sl a & \sl b & \sl c & \sl d & \sl e & \sl f & \sl g 
           & \sl h & \sl i & \sl j & \sl k & \sl l & \sl m & \sl n 
           & \sl o & \sl p & \sl q & \sl r & \sl s & \sl t & \sl u 
           & \sl v & \sl w & \sl x & \sl y & \sl z \\
\hline
\hline
\multicolumn{27}{l}{} \\
\hline
\hline
typewriter & \tt A & \tt B & \tt C & \tt D & \tt E & \tt F & \tt G 
           & \tt H & \tt I & \tt J & \tt K & \tt L & \tt M & \tt N 
           & \tt O & \tt P & \tt Q & \tt R & \tt S & \tt T & \tt U 
           & \tt V & \tt W & \tt X & \tt Y & \tt Z \\
\verb|{\tt ...}| 
           & \tt a & \tt b & \tt c & \tt d & \tt e & \tt f & \tt g 
           & \tt h & \tt i & \tt j & \tt k & \tt l & \tt m & \tt n 
           & \tt o & \tt p & \tt q & \tt r & \tt s & \tt t & \tt u 
           & \tt v & \tt w & \tt x & \tt y & \tt z \\
\hline
sans serif & \sf A & \sf B & \sf C & \sf D & \sf E & \sf F & \sf G 
           & \sf H & \sf I & \sf J & \sf K & \sf L & \sf M & \sf N 
           & \sf O & \sf P & \sf Q & \sf R & \sf S & \sf T & \sf U 
           & \sf V & \sf W & \sf X & \sf Y & \sf Z \\
\verb|{\sf ...}| 
           & \sf a & \sf b & \sf c & \sf d & \sf e & \sf f & \sf g 
           & \sf h & \sf i & \sf j & \sf k & \sf l & \sf m & \sf n 
           & \sf o & \sf p & \sf q & \sf r & \sf s & \sf t & \sf u 
           & \sf v & \sf w & \sf x & \sf y & \sf z \\
\hline
small caps & \sc A & \sc B & \sc C & \sc D & \sc E & \sc F & \sc G 
           & \sc H & \sc I & \sc J & \sc K & \sc L & \sc M & \sc N 
           & \sc O & \sc P & \sc Q & \sc R & \sc S & \sc T & \sc U 
           & \sc V & \sc W & \sc X & \sc Y & \sc Z \\
\verb|{\sc ...}| 
           & \sc a & \sc b & \sc c & \sc d & \sc e & \sc f & \sc g 
           & \sc h & \sc i & \sc j & \sc k & \sc l & \sc m & \sc n 
           & \sc o & \sc p & \sc q & \sc r & \sc s & \sc t & \sc u 
           & \sc v & \sc w & \sc x & \sc y & \sc z \\
\hline
\hline
\multicolumn{27}{l}{} \\
\hline
\hline
math       & $A$&$B$&$C$&$D$&$E$&$F$&$G$&$H$&$I$&$J$&$K$&$L$&$M$&$N$
            &$O$&$P$&$Q$&$R$&$S$&$T$&$U$&$V$&$W$&$X$&$Y$&$Z$ \\
\verb|$...$| 
           & $a$&$b$&$c$&$d$&$e$&$f$&$g$&$h$&$i$&$j$&$k$&$l$&$m$&$n$
            &$o$&$p$&$q$&$r$&$s$&$t$&$u$&$v$&$w$&$x$&$y$&$z$ \\
\hline
fraktur    & $\mathfrak A$ & $\mathfrak B$ & $\mathfrak C$ & $\mathfrak D$ 
           & $\mathfrak E$ & $\mathfrak F$ & $\mathfrak G$ & $\mathfrak H$ 
           & $\mathfrak I$ & $\mathfrak J$ & $\mathfrak K$ & $\mathfrak L$ 
           & $\mathfrak M$ & $\mathfrak N$ & $\mathfrak O$ & $\mathfrak P$ 
           & $\mathfrak Q$ & $\mathfrak R$ & $\mathfrak S$ & $\mathfrak T$ 
           & $\mathfrak U$ & $\mathfrak V$ & $\mathfrak W$ & $\mathfrak X$ 
           & $\mathfrak Y$ & $\mathfrak Z$ \\
\verb|$\mathfrak{...}$| 
           & $\mathfrak a$ & $\mathfrak b$ & $\mathfrak c$ & $\mathfrak d$ 
           & $\mathfrak e$ & $\mathfrak f$ & $\mathfrak g$ & $\mathfrak h$ 
           & $\mathfrak i$ & $\mathfrak j$ & $\mathfrak k$ & $\mathfrak l$ 
           & $\mathfrak m$ & $\mathfrak n$ & $\mathfrak o$ & $\mathfrak p$ 
           & $\mathfrak q$ & $\mathfrak r$ & $\mathfrak s$ & $\mathfrak t$ 
           & $\mathfrak u$ & $\mathfrak v$ & $\mathfrak w$ & $\mathfrak x$ 
           & $\mathfrak y$ & $\mathfrak z$ \\
\hline
calligraphic  
           & $\mathcal A$ & $\mathcal B$ & $\mathcal C$ & $\mathcal D$ 
           & $\mathcal E$ & $\mathcal F$ & $\mathcal G$ & $\mathcal H$ 
           & $\mathcal I$ & $\mathcal J$ & $\mathcal K$ & $\mathcal L$ 
           & $\mathcal M$ & $\mathcal N$ & $\mathcal O$ & $\mathcal P$ 
           & $\mathcal Q$ & $\mathcal R$ & $\mathcal S$ & $\mathcal T$ 
           & $\mathcal U$ & $\mathcal V$ & $\mathcal W$ & $\mathcal X$ 
           & $\mathcal Y$ & $\mathcal Z$ \\
\verb|$\mathcal{...}$| 
           & \multicolumn{26}{|l|}{\it (no lowercase calligraphic)} \\
\hline
blackboard bold
           & $\mathbb A$ & $\mathbb B$ & $\mathbb C$ & $\mathbb D$ 
           & $\mathbb E$ & $\mathbb F$ & $\mathbb G$ & $\mathbb H$ 
           & $\mathbb I$ & $\mathbb J$ & $\mathbb K$ & $\mathbb L$ 
           & $\mathbb M$ & $\mathbb N$ & $\mathbb O$ & $\mathbb P$ 
           & $\mathbb Q$ & $\mathbb R$ & $\mathbb S$ & $\mathbb T$ 
           & $\mathbb U$ & $\mathbb V$ & $\mathbb W$ & $\mathbb X$ 
           & $\mathbb Y$ & $\mathbb Z$ \\
\verb|$\mathbb{...}$| 
           & \multicolumn{26}{|l|}{\it (no lowercase blackboard bold)} \\
\hline
\hline
\multicolumn{27}{l}{} \\
\hline
\hline
\verb|{\tiny ...}|  
   & \multicolumn{26}{|l|}{\tiny Every good boy likes squid.} \\
\hline
\verb|{\small ...}| 
   & \multicolumn{26}{|l|}{\small Every good boy likes squid.} \\
\hline
\verb|{\large ...}| 
   & \multicolumn{26}{|l|}{\large Every good boy likes squid.} \\
\hline
\verb|{\Large ...}| 
   & \multicolumn{26}{|l|}{\Large Every good boy likes squid.} \\
\hline
\verb|{\huge ...}|  
   & \multicolumn{26}{|l|}{\huge Every good boy likes squid.} \\
\hline
\verb|{\Huge ...}|  
   & \multicolumn{26}{|l|}{\Huge Every good boy likes squid.} \\
\hline
\hline
\stoptable


\section*{Remarks}

\begin{itemize}

\item Text fonts can also be called by \verb|\textbf{...}|, 
\verb|\textit{...}| (which does automatic italic correction), 
{\it et cetera}.  

\item Fonts in the math environment can be called using
\verb|\mathrm{...}|, \verb|\mathit{...}|, and so on.  

\item Some simple style combinations are possible, but sometimes 
the order matters, and some combinations may not work without 
resorting to different font commands.  For example, 
\verb|{\it \bf Squid}| produces only the bold {\it \bf Squid}, 
\verb|{\bf \it Squid}| produces only the italic {\bf \it Squid},
\verb|{\bf \itshape Squid}| produces the bold, italic {\bf \itshape Squid}.

\end{itemize}

%------------------------------------------------------------------
% PROBLEM, PART, AND POINT COUNTING...

% Create the problem number counter.  Initialize to zero.
\newcounter{problemnum}

% Specify that problems should be labeled with arabic numerals.
\renewcommand{\theproblemnum}{\arabic{problemnum}}


% Create the part-within-a-problem counter, "within" the problem counter.
% This counter resets to zero automatically every time the PROBLEMNUM counter
% is incremented.
\newcounter{partnum}[problemnum]

% Specify that parts should be labeled with lowercase letters.
\renewcommand{\thepartnum}{\alph{partnum}}

% Make a counter to keep track of total points assigned to problems...
\newcounter{totalpoints}

% Make counters to keep track of points for parts...
\newcounter{curprobpts}		% Points assigned for the problem as a whole.
\newcounter{totalparts}		% Total points assigned to the various parts.

% Make a counter to keep track of the number of points on each page...
\newcounter{pagepoints}
% This counter is reset each time a page is printed.

% This "program" keeps track of how many points appear on each page, so that
% the total can be printed on the page itself.  Points are added to the total
% for a page when the PART (not the problem) they are assigned to is specified.
% When a problem without parts appears, the PAGEPOINTS are incremented directly
% from the problem as a whole (CURPROBPTS).


%---------------------------------------------------------------------------


% The \problem environment first checks the information about the previous
% problem.  If no parts appeared (or if they were all assigned zero points,
% then it increments TOTALPOINTS directly from CURPROBPTS, the points assigned
% to the last problem as a whole.  If the last problem did contain parts, it
% checks to make sure that their point values total up to the correct sum.
% It then puts the problem number on the page, along with the points assigned
% to it.

\newenvironment{problem}[1]{
% STATEMENTS TO BE EXECUTED WHEN A NEW PROBLEM IS BEGUN:
%
% Increment the problem number counter, and set the current \ref value to that
% number.
\refstepcounter{problemnum}
%
% Add some vspace to separate from the last problem.
\vspace{0.15in} \par
%
\setcounter{curprobpts}{#1} \setcounter{totalparts}{0}	% Reset counters.
%
% Now put in the "announcement" on the page.
{\Large \bf \theproblemnum. \normalsize %({\it \arabic{curprobpts}
%{point\null\ifnum \value{curprobpts} = 1\else s\fi}\/)
}
}{
% STATEMENTS TO BE EXECUTED WHEN AN OLD PROBLEM IS ENDED:
%
% If no parts to problem, then increment TOTALPOINTS and PAGEPOINTS for the
% entire problem at once.
\ifnum \value{totalparts} = 0
	\addtocounter{totalpoints}{\value{curprobpts}}	% Add pts to total.
	\addtocounter{pagepoints}{\value{curprobpts}}	% Add pts to page total.
%
% If there were parts for the problem, then check to make sure they total up
% to the same number of points that the problem is worth. Issue a warning
% if not.
\else \ifnum \value{totalparts} = \value{curprobpts}
	\else \typeout{}
	\typeout{!!!!!!!   POINT ACCOUNTING ERROR   !!!!!!!!}
	\typeout{PROBLEM [\theproblemnum] WAS ALLOCATED \arabic{curprobpts} POINTS,}
	\typeout{BUT CONTAINS PARTS TOTALLING \arabic{totalparts} POINTS!}
	\typeout{}
	\fi
\fi
}


%---------------------------------------------------------------------------


% The \newpart command increments the part counter and displays an appropriate
% lowercase letter to mark the part.  It adds points to the point counter
% immediately.  If 0 points are specified, no point announcement is made.
% Otherwise, the announcement is in scriptsize italics.

\newcommand{\newpart}[1]
{
\refstepcounter{partnum}	% Set the current \ref value to the part number.
\hspace{0.25in}		% Indent the part by a quarter inch.
%
% If points are to be printed for this problem (signaled by point value > 0),
% then put them in in scriptsize italics.
\ifnum #1 > 0
	\makebox[0.5in][l]{{\bf \thepartnum.} {\bf ({\it #1 pt\ifnum #1 = 1\else s\fi\/}) \,\,}}
\else
	\makebox[0.25in][l]{({\bf \thepartnum})}
\fi
%
\hspace{0.1in}		% Lead the material away from the part "number".
%
\addtocounter{totalparts}{#1}	% Add points to totalparts for this problem.
\addtocounter{pagepoints}{#1}	% Add points to total for this page.
\addtocounter{totalpoints}{#1}	% Add points to total for entire test.
}


%---------------------------------------------------------------------------



% Just in case you want to skip some numbers in your test...

\newcommand{\skipproblem}[1]{\addtocounter{problemnum}{#1}}



%---------------------------------------------------------------------------


% The \showpoints command simply gives a count of the total points read in up to
% the location at which the command is placed.  Typically, one places one
% \showpoints command at the end of the latex file, just prior to the
% \end{document} command.  It can appear elsewhere, however.

\newcommand{\showpoints}
{
\typeout{}  
\typeout{====> A TOTAL OF \arabic{totalpoints} POINTS WERE READ.}
\typeout{}
}


%---------------------------------------------------------------------------



\leftline{Pat Q.~Student}
\leftline{AME 20231}
\leftline{9 February 2021}

\medskip
This is a sample file in the text formatter \LaTeX.
I require you to use it for the following reasons:

\begin{itemize}

\item{It produces the best output of text, figures,
      and equations of any
      program I've seen.}

\item{It is machine-independent. It runs on Linux, Macintosh (see {\tt TeXShop}), and Windows (see {\tt MiKTeX}) machines.  There are web-based versions, \href{https://www.overleaf.com}{\tt https://www.overleaf.com}.
     You can e-mail {\tt ASCII} text versions of most relevant files.}

\item{It is the tool of choice for many research
     scientists and engineers.
     Many journals accept 
     \LaTeX~ submissions, and many books
     are written in \LaTeX.}

\end{itemize}
\medskip
Some basic instructions are given next.
Put your text in here.  You can be a little sloppy    about
spacing.  It adjusts the text to look good.
{\small You can make the text smaller.}
{\tiny You can make the text tiny.}

Skip a line for a new paragraph.   
You can use italics ({\em e.g.} {\em  Thermodynamics is everywhere}) or {\bf bold}.
Greek letters are a snap: $\Psi$, $\psi$,
$\Phi$, $\phi$.  Equations within text are easy---
A well known Maxwell thermodynamic relation is
$\left.{\partial T \over \partial P}\right|_{s} = 
\left.{\partial v \over \partial s}\right|_{P}$.
You can also set aside equations like so:
\begin{eqnarray}
du &=& T\ ds -P\ dv, \qquad \mbox{first law.}\label{fl}\\
ds &\ge& {\delta q \over T}.\qquad  \qquad \mbox{second law.} \label{sl}
\end{eqnarray}
Eq.~(\ref{fl}) is the first law.
Eq.~(\ref{sl}) is the second law.
References\footnote{Lamport, L., 1986, {\em \LaTeX: User's Guide \& Reference Manual},
    Addison-Wesley: Reading, Massachusetts.}
are available. 
If you have an postscript file, say {\tt sample.figure.eps}, in the same local directory,
you can insert the file as a figure.  Figure \ref{sample}, below, plots an isotherm for air modeled as an ideal gas. 
\begin{figure}[ht]
\epsfxsize=2.5in
\centerline{\epsffile{sample.figure.eps}}
\caption{Sample figure plotting $T=300~{\rm K}$ isotherm for air when modeled as an ideal gas.}
\label{sample}
\end{figure}

\medskip
\leftline{\em Running \LaTeX}
\medskip

You can create a \LaTeX~ file with any text editor ({\tt vi}, {\tt emacs}, {\tt gedit}, 
etc.). 
To get a document, you need to run the \LaTeX~ application
on the text file.  The text file must have the suffix ``{\tt .tex}''
On a Linux cluster machine, this is done via the command

\medskip
{\tt latex file.tex}

\medskip
\noindent
This generates three files: {\tt file.dvi}, {\tt file.aux},
and {\tt file.log}.  The most important is {\tt file.dvi}. 

\medskip
\noindent
The finished product can be previewed in the following way.
Execute the commands:

\medskip

{\tt dvipdf file.dvi}\hspace{1.9in}{\em Linux System}

\medskip
\noindent
This command generates {\tt file.pdf}.  
Alternatively, you can use {\tt TeXShop} on a Macintosh or {\tt MiKTeX} on a Windows-based machine. {\em Another very good and modern option is the web-based} \href{https://www.overleaf.com}{\tt https://www.overleaf.com}.
The {\tt .tex} file must have a closing statement as
below.

\section*{Notes for My Paper}

Don't forget to include examples of topicalization.
They look like this:

{\small
\enumsentence{Topicalization from sentential subject:\\ 
\shortex{7}{a John$_i$ [a & kltukl & [el & 
  {\bf l-}oltoir & er & ngii$_i$ & a Mary]]}
{ & {\bf R-}clear & {\sc comp} & 
  {\bf IR}.{\sc 3s}-love   & P & him & }
{John, (it's) clear that Mary loves (him).}}
}

\subsection*{How to handle topicalization}

I'll just assume a tree structure like (\ex{1}).

{\small
\enumsentence{Structure of A$'$ Projections:\\ [2ex]
\begin{tabular}[t]{cccc}
    & \node{i}{CP}\\ [2ex]
    \node{ii}{Spec} &   &\node{iii}{C$'$}\\ [2ex]
        &\node{iv}{C} & & \node{v}{SAgrP}
\end{tabular}
\nodeconnect{i}{ii}
\nodeconnect{i}{iii}
\nodeconnect{iii}{iv}
\nodeconnect{iii}{v}
}
}

%%%(change to appropriate class and semester)
26L-04 Spring 1999 

%%%(change to appropriate date)
Gateway XX.XX.XX \hfill {Name:} {\underline {\hspace{2.5in}}}
\vspace{2pc}

%%%(modify rules, time, points as appropriate)
You must get sevenof the eight problems completely correct to pass, 
so be sure to check your answers a couple of times.  Circle your 
final answers (work will not be graded or looked at). Good luck!

\vspace{2pc}

For problems 1-7 below, find $\displaystyle\frac{dy}{dx}$.

\begin{problem}{0}
$\displaystyle y(x)= $
\end{problem}
\vfill

\begin{problem}{0}
$\displaystyle y(x)= $
\end{problem}
\vfill

\begin{problem}{0}
$\displaystyle y(u)= $ 
\hspace{.2 in} $\displaystyle u(x)= $
\end{problem}
\vfill

\begin{problem}{0}
$\displaystyle y(x)= $
\end{problem}
\vfill

\begin{problem}{0}
$\displaystyle y(x)= $
\end{problem}
\vfill

\begin{problem}{0}
$\displaystyle y(x)= $
\end{problem}
\vfill

\begin{problem}{0}
$\displaystyle y(x)= $
\end{problem}
\vfill

For problem 8, find the second derivative of $y(x)$.

\begin{problem}{0}
$\displaystyle y(x)= $
\end{problem}
\vfill



% ooooooooooooooooooooooooooooooooooooooooooooooooooooooooooooooo
%                             COVER
% ooooooooooooooooooooooooooooooooooooooooooooooooooooooooooooooo 

\centerline{\huge \bf TEST XXX}        %%%(number of test)
\vfill \vfill
   
Math XXX                               %%%(class number and section) 

XXX \hfill                             %%%(date of test)
{\bf Name: } $\underbrace{\hspace{2.7in}}_{\mbox{\tiny by writing my 
                                           name i swear by the honor code}}$
\vfill \vfill \vfill

{\bf Read all of the following information before starting the exam:}
\vspace{1pc}

\begin{itemize}                        %%%(change info. as desired)
	\item  Show all work, clearly and in order, if you want to get full
	credit.  I reserve the right to take off points if I cannot see how you 
	arrived at your answer (even if your final answer is correct).
	
	\item Justify your answers algebraically whenever possible to ensure 
	full credit. When you do use your calculator, sketch all relevant 
	graphs and explain all relevant mathematics.
	
	\item  Circle or otherwise indicate your final answers.

	\item  Please keep your written answers brief; be clear and to the point.
	I will take points off for rambling and for incorrect or irrelevant 
	statements.
		
	\item  This test has XXX problems  %%%(number of problems)
	and is worth XXX points,           %%%(insert total number of points)  
	plus some extra credit at the end.  It is your responsibility to 
	make sure that you have all of the pages!
	
	\item  Good luck!
\end{itemize}

\vfill \vfill \vfill

\clearpage


% ooooooooooooooooooooooooooooooooooooooooooooooooooooooooooooooo
%                           PAGE ONE
% ooooooooooooooooooooooooooooooooooooooooooooooooooooooooooooooo


% problem with no parts
\begin{problem}{0}
PROBLEM HERE    
\vfill \vfill
\end{problem}


% problem with three parts
\begin{problem}{0}
START OF PROBLEM
\vspace{1pc}

\newpart{0}
PART A
\vfill

\newpart{0}
PART B
\vfill

\newpart{0}
PART C
\vfill
\end{problem}


\clearpage


% ooooooooooooooooooooooooooooooooooooooooooooooooooooooooooooooo
%                           PAGE TWO
% ooooooooooooooooooooooooooooooooooooooooooooooooooooooooooooooo


% problem with equation
\begin{problem}{0}
THIS PROBLEM CONCERNS THE EQUATION

$$ \frac{dy}{dx} = xy + 1 $$

\vfill
\end{problem}


% problem with graph
\begin{problem}{0}
THIS PROBLEM CONCERNS THE GRAPH

\begin{center}
\epsfig{file=graph.eps,height=2in,width=2.25in}   
\end{center}

\vfill
\end{problem}


\clearpage


% ooooooooooooooooooooooooooooooooooooooooooooooooooooooooooooooo
%                           PAGE THREE
% ooooooooooooooooooooooooooooooooooooooooooooooooooooooooooooooo


% problem with horizontal table
\begin{problem}{0}
THIS PROBLEM CONCERNS THE TABLE

\begin{table}[h] 
\begin{center}
\addtolength{\tabcolsep}{1mm}
\renewcommand{\arraystretch}{1.2}
\begin{tabular}{|c|r|r|r|r|r|}
\hline
\hline
$t$    & \phantom{0}1.0 & \phantom{0}1.2 
                   &  1.4 &  1.6 &  1.8 \\
\hline
$f(t)$ & 3.0 & 7.0 & 10.0 & 12.0 & 13.0 \\
\hline
\hline
\end{tabular}
\end{center}
\end{table}

\vfill
\end{problem}


% problem with vertical table
\begin{problem}{0}
THIS PROBLEM CONCERNS THE TABLE

\begin{table}[h] 
\begin{center}
\addtolength{\tabcolsep}{4mm}
\renewcommand{\arraystretch}{1.2}
\begin{tabular}{|c|r|r|}
\hline
$x$ & \multicolumn{1}{|c|}{$f(x)$} & \multicolumn{1}{|c|}{$g(x)$} \\
\hline
\hline
1 & 20   & 100 \\
2 & 113  & 447 \\
3 & 311  & 649 \\
4 & 640  & 793 \\
5 & 1118 & 904 \\
6 & 1764 & 996 \\
\hline
\end{tabular}
\end{center}
\end{table}

\vfill
\end{problem}


\clearpage


% ooooooooooooooooooooooooooooooooooooooooooooooooooooooooooooooo
%                           PAGE FOUR
% ooooooooooooooooooooooooooooooooooooooooooooooooooooooooooooooo





\clearpage


% ooooooooooooooooooooooooooooooooooooooooooooooooooooooooooooooo
%                           PAGE FIVE
% ooooooooooooooooooooooooooooooooooooooooooooooooooooooooooooooo





\clearpage


% ooooooooooooooooooooooooooooooooooooooooooooooooooooooooooooooo
%                           PAGE SIX
% ooooooooooooooooooooooooooooooooooooooooooooooooooooooooooooooo





\clearpage


% ooooooooooooooooooooooooooooooooooooooooooooooooooooooooooooooo
%                           PAGE SEVEN
% ooooooooooooooooooooooooooooooooooooooooooooooooooooooooooooooo





\clearpage


% ooooooooooooooooooooooooooooooooooooooooooooooooooooooooooooooo
%                           PAGE EIGHT
% ooooooooooooooooooooooooooooooooooooooooooooooooooooooooooooooo





\clearpage


% ooooooooooooooooooooooooooooooooooooooooooooooooooooooooooooooo
%                         BONUS & SURVEY
% ooooooooooooooooooooooooooooooooooooooooooooooooooooooooooooooo

{\large \bf Bonus Question (2 Extra Credit Points):}
\vspace{1pc} \vspace{1pc}

BONUS QUESTION  
\vfill

{\large \bf Survey Question (2 Extra Credit Points):}
\vspace{1pc} \vspace{1pc}

SURVEY QUESTION  
\vfill

\clearpage

% ooooooooooooooooooooooooooooooooooooooooooooooooooooooooooooooo
%                             SCRAP
% ooooooooooooooooooooooooooooooooooooooooooooooooooooooooooooooo 

{\large \bf Scrap Page}

(please do not remove this page from the test packet)



\showpoints

\subsection*{Mood}

Mood changes when there is a topic, as well as when
there is WH-movement.  \emph{Irrealis} is the mood when
there is a non-subject topic or WH-phrase in Comp.
\emph{Realis} is the mood when there is a subject topic
or WH-phrase.

\end{document}